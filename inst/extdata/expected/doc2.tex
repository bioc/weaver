%
% NOTE -- ONLY EDIT THE .Rnw FILE!!!  The .tex file is
% likely to be overwritten.
%
\documentclass[11pt]{article}

\usepackage{times}
\usepackage{hyperref}

\usepackage[authoryear,round]{natbib}
\usepackage{times}
\usepackage{comment}

\textwidth=6.2in
\textheight=8.5in
%\parskip=.3cm
\oddsidemargin=.1in
\evensidemargin=.1in
\headheight=-.3in


\newlength{\smallfigwidth}
\setlength{\smallfigwidth}{6cm}

\newcommand{\scscst}{\scriptscriptstyle}
\newcommand{\scst}{\scriptstyle}

\newcommand{\Rfunction}[1]{{\texttt{#1}}}
\newcommand{\Robject}[1]{{\texttt{#1}}}
\newcommand{\Rpackage}[1]{{\textsf{#1}}}
\newcommand{\Rclass}[1]{{\textit{#1}}}

\newcommand{\MAT}[1]{{\bf #1}}
\newcommand{\VEC}[1]{{\bf #1}}

\newcommand{\Amat}{{\MAT{A}}}

%%notationally this is going to break
\newcommand{\Emat}{{\MAT{E}}}
\newcommand{\Xmat}{{\MAT{X}}}
\newcommand{\Xvec}{{\VEC{X}}}
\newcommand{\xvec}{{\VEC{x}}}


\newcommand{\Zvec}{{\VEC{Z}}}
\newcommand{\zvec}{{\VEC{z}}}

\newcommand{\calG}{\mbox{${\cal G}$}}

\bibliographystyle{plainnat}

\title{Test Document for weaver}
\author{S. Falcon}

\usepackage{/Users/seth/proj/builds/R-2.4/share/texmf/Sweave}
\begin{document}

\maketitle

\section{Introduction}

This is doc2.  We'll reuse some chunks with a few changes and pickup
the values that were cached during the processing of doc1.

\begin{Schunk}
\begin{Sinput}
> set.seed(123)
> b <- rnorm(3)
> firstB <- b
> c <- runif(3) * 2
> z <- b + c
> z
\end{Sinput}
\end{Schunk}


\begin{Schunk}
\begin{Sinput}
> firstB
\end{Sinput}
\begin{Soutput}
[1] -0.5604756 -0.2301775  1.5587083
\end{Soutput}
\begin{Sinput}
> b
\end{Sinput}
\begin{Soutput}
[1] 2.112422 3.569676 2.205740
\end{Soutput}
\begin{Sinput}
> c
\end{Sinput}
\begin{Soutput}
[1] 1.056211 1.784838 1.102870
\end{Soutput}
\begin{Sinput}
> z
\end{Sinput}
\begin{Soutput}
[1] 0.4957353 1.5546606 2.6615783
\end{Soutput}
\end{Schunk}

\begin{Schunk}
\begin{Sinput}
> aStr <- "Strings can be cached"
> bStr <- paste(aStr, 1:3, sep = "--->")
\end{Sinput}
\end{Schunk}

An unnamed code chunk:

\begin{Schunk}
\begin{Sinput}
> bbbb <- 10
\end{Sinput}
\end{Schunk}

Another unnamed chunk

\begin{Schunk}
\begin{Sinput}
> xyz <- bbbb + 1
> print(xyz)
\end{Sinput}
\begin{Soutput}
[1] 11
\end{Soutput}
\begin{Sinput}
> print(bStr)
\end{Sinput}
\begin{Soutput}
[1] "Strings can be cached--->1" "Strings can be cached--->2"
[3] "Strings can be cached--->3"
\end{Soutput}
\end{Schunk}

\end{document}
